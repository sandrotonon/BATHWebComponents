\chapter{Web Components nach dem vorläufigen W3C-Standard}\label{web-components-nach-w3c}

In diesem Kapitel wird in Abschnitt \ref{problemloesung} auf die Problemlösung der Web Components nach den Vorstellungen des \ac{W3C}s eingegangen. In Abschnitt \ref{custom-elements} wird die erste Technologie vorgestellt, die Custom Elements, Abschnitt \ref{html-templates} wird sich den \ac{HTML} Templates widmen, in Abschnitt \ref{shadow-dom} wird auf den Shadow \ac{DOM} eingegangen und in Abschnitt \ref{html-imports} die letzte Technologie, die \ac{HTML} Imports, gezeigt. In Abschnitt \ref{polyfills-mit-webcomponents.js} werden die für diese Technologien entwickelten Polyfills erklärt. Abschließend wird in Kapitel \ref{implementierung-einer-komponente-mit-den-nativen-web-component-apis} eine exemplarische Komponente implementiert.


\section{Problemlösung}\label{problemloesung}

In der heutigen Webentwicklung kommt es häufig vor, dass für diverse Probleme oftmals die gleiche, oder eine ähnliche Lösung entwickelt werden muss. Diese unterscheiden sich stets leicht, bringen im Kern aber dennoch meist dieselben Features mit sich. Um diese Features auf der Webseite verfügbar zu machen, sind eine Reihe an verschiedenen Technologien notwendig. Zum einen muss das von ihr benötigte \ac{HTML}-Markup geschrieben werden, zum anderen muss ein JavaScript eingebunden und über eine vorgegebene \ac{API} konfiguriert werden. Damit die Komponente auch optisch funktioniert, muss ein entsprechendes Style\-sheet mit den Style-Definitionen eingebunden werden. Da \ac{CSS}-Regeln immer global auf das gesamte Dokument angewendet werden, kann es dabei zu ungewollten Auswirkungen auf andere Bestandteile der Webseite kommen. Fasst man diese Punkte zusammen, so wird deutlich, dass es in der Webentwicklung kein Plugin-System gibt, um Webseiten schnell und einfach zu erweitern.

Diesem Problem widmen sich die Web Components. Sie sollen der Frontend-Ent\-wick\-lung ein Plugin-System bereitstellen, welches diese Probleme löst. Eine Komponente steht dabei als eigenes \ac{HTML}-Element, welches ihre gesamte Funktionalität kapselt und nach außen verbirgt. Konflikte mit anderen Komponenten oder der einbindenden Webseite selbst werden somit vermieden. Dabei ist das Verhalten nach außen für jede Komponente dasselbe, es gibt also für jede Komponente die gleiche Schnittstelle, um sie zu konfigurieren und einzubinden. Dies erleichtert deren Umgang deutlich, da die einzige benötigte Technologie \ac{HTML} ist. Dadurch können Komponenten verwendet werden wie jedes native \ac{HTML}-Element. Sie sind verschachtelbar und haben Attribute, über welche sie konfiguriert werden können. Web Components bilden dabei eine Sammlung an Technologien, um jene Eigenschaften zu gewährleisten. In den folgenden Abschnitten werden diese Technologien erklärt und auf ihre Anwendung eingegangen.
