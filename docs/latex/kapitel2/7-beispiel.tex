\section{Implementierung einer Komponente mit den nativen Web-Component-APIs}\label{implementierung-einer-komponente-mit-den-nativen-web-component-apis}

Anhand der vorhergehenden Abschnitte wird in diesem Abschnitt die Implementierung der Komponente \texttt{\textless{}custom-element\textgreater{}} mit den nativen \ac{HTML} APIs erläutert. Diese soll dabei das Markup in einem Shadow \ac{DOM} kapseln und den übergebenen Inhalt darstellen. Des Weiteren soll dessen Farbe über das Attribut \texttt{theme} optional konfiguriert werden können.


\subsection{Custom Element mit Eigenschaften und Funktionen definieren}\label{custom-element-mit-eigenschaften-und-funktionen-definieren}

Um ein neues Custom Element zu registrieren, wird zunächst ein \texttt{\ac{HTML}Element} Prototyp \texttt{CustomElementProto} mittels \texttt{Object.create(\ac{HTML}Element.prototype)} erstellt. Dieser wird anschließend um die Eigenschaft \texttt{theme} und dessen Standardwert \texttt{style1} erweitert, welches das deklarativ konfigurierbare
Attribut \texttt{theme} abbildet. Nun können die Lifecycle-Callback-Funktionen \texttt{createdCallback} und \texttt{attributeChangedCallback} der Komponente definiert werden (siehe Listing \ref{cemeufd}).

\lstinputlisting[language=JavaScript,label=cemeufd,caption=Custom Element mit Eigenschaften und Funktionen definieren]{kapitel2/listings/7-1.js}

Die \texttt{createdCallback}-Funktion soll zunächst prüfen ob das Attribut \texttt{theme} beim Verwenden des
\texttt{\textless{}custom-element\textgreater{}}-Tags verwendet und ein entsprechender Wert gesetzt wurde und übergibt dieses der Hilfsfunktion \texttt{setTheme}. Wird das Attribut nicht gesetzt, wird der Standardwert \texttt{style1} übergeben. Falls das \texttt{style}-Attribut von Außen geändert wird, soll die \texttt{attributeChangedCallback} Funktion gewährleisten, dass die Änderung auch von der Komponente übernommen wird, indem sie das Attribut der Hilfsfunktion \texttt{setTheme} übergibt. Um das Setzen und Ändern des \texttt{theme}-Attributs zu implementieren wird zuletzt die Hilfsfunktion \texttt{setTheme} für den Prototyp definiert (siehe Listing \ref{hsetthemed}).

\lstinputlisting[language=JavaScript,label=hsetthemed,caption=HilfsFunktion \texttt{setTheme} definieren]{kapitel2/listings/7-2.js}

Diese setzt den übergebenen Parameter, das \texttt{theme}-Attribut, als Klasse auf den umschließenden Wrapper \texttt{.outer}, welche dabei den zu verwendenden Style der Komponente bestimmt. Da der Prototyp nun alle erforderlichen Eigenschaften besitzt, kann er mit \texttt{document.registerElement(\dq custom-element\dq,\ \{\ prototype:\ CustomElementProto\ \});} als \ac{HTML}-Tag \texttt{custom-element} in dem importierenden Dokument verfügbar gemacht werden.


\subsection{Template erstellen und Styles definieren}\label{template-erstellen-und-styles-definieren}

Bisher ist das Custom Element zwar funktional, bietet aber noch kein gekapseltes Markup. Hierfür wird ein Template mit der ID \texttt{myElementTemplate} angelegt (siehe Listing \ref{teusd}), welches die für die Komponente notwendige \ac{HTML} Struktur beinhaltet.

\lstinputlisting[language=HTML,label=teusd,caption=Template erstellen und Styles definieren]{kapitel2/listings/7-3.html}

Dieses enthält dabei einen Insertion Point \texttt{\textless{}content\textgreater{}}, in welchem die Kind-Elemente der Komponente in das interne Markup projiziert werden. Zusätzlich werden 2 Hilfs-Wrapper und Text definiert, damit die Elemente schneller mittels JavaScript selektierbar sind und das gewünschte Aussehen erreicht wird. Um nun die verschiedenen, mittels dem \texttt{theme}-Attribut konfigurierbaren, Styles zur Verfügung zu stellen, werden diese neben generellen Styles in einem \texttt{\textless{}style\textgreater{}}-Tag in dem Template definiert (siehe Listing \ref{stylesdefinieren}). In diesem Beispiel werden 2 Optionen, \texttt{style1} und \texttt{style2} zur Verfügung gestellt.

\lstinputlisting[language=HTML,label=stylesdefinieren,caption=Styles definieren]{kapitel2/listings/7-4.html}

Das Template wird nun zwar schon heruntergeladen, jedoch noch nicht in den \ac{DOM} eingefügt. Hierzu muss es dem Shadow Root hinzugefügt werden, was in dem nachfolgenden Abschnitt \ref{template-bereitstellen-und-shadow-dom-zur-kapselung-benutzen} dargestellt wird.


\subsection{Template bereitstellen und Shadow DOM zur Kapselung benutzen}\label{template-bereitstellen-und-shadow-dom-zur-kapselung-benutzen}

Bevor das erstellte Template eingebunden werden kann, muss zunächst ein Shadow Root mittels \texttt{var\ shadow\ =\ this.createShadowRoot();} erzeugt werden. Hierfür wird die bereits definierte Lifecycle-Callback-Funktion \texttt{createdCallback} erweitert. Somit kann der Shadow \ac{DOM} sofort initialisiert werden, wenn das Element erzeugt wurde. Nun kann der Inhalt des Templates mit der ID \texttt{myElementTemplate} mittels \texttt{var\ template\ =\ importDoc.querySelector('\#myElementTemplate').content;} importiert und mit der Anweisung \texttt{shadow.appendChild(template.cloneNode(true));} dem Shadow Root hinzugefügt werden. Die Variable \texttt{importDoc} stellt dabei die Referenz auf die importierte Komponente, also das \texttt{\textless{}custom-element\textgreater{}}-Element, dar und kann mittels der Funktion \texttt{var\ importDoc\ =\ document.currentScript.ownerDocument;} ermittelt werden. Wird dies nicht getan, so würde der \texttt{querySelector} auf das Eltern-Dokument der eingebetteten Komponente zugreifen und das Template nicht finden. Nun ist der Inhalt des Templates als Shadow \ac{DOM} innerhalb des Elements gekapselt und nach Außen nicht sichtbar. Die gerenderte Komponente wird in Abbildung \ref{fig:gwkmnapis} dargestellt.

\begin{figure}[htbp]
 \centering
 \includegraphics[width=5cm,keepaspectratio]{kapitel2/bilder/7-beispiel}
 \caption{Gerenderte Komponente mit nativen APIs}
 \label{fig:gwkmnapis}
\end{figure}


\subsection{Element importieren und verwenden}\label{element-importieren-und-verwenden}

Das Element ist somit vollständig und kann in einer beliebigen Webseite oder Applikation eingesetzt werden. Hierzu muss das Element mittels \texttt{\textless{}link\ rel=\dq import\dq\ href=\dq elements/custom-element.html\dq\textgreater{}} zunächst importiert werden. Es kann anschließend mit entsprechenden Attributen und Inhalt auf der Seite eingebettet werden, wie beispielsweise der Konfiguration \texttt{\textless{}custom-element\ theme=\dq style1\dq\textgreater{}Light DOM\textless{}/custom-element\textgreater{}}. Das vollständige Beispiel der Komponente, sowie dessen Einbindung in ein \ac{HTML}-Dokument sind im Anhang A zu finden.
