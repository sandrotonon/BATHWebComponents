\setcounter{tocdepth}{2}  %% Uebreschriften bis subsectionw ins Inhaltsverzeichnis
\setcounter{secnumdepth}{2}  %% Nummerierung bis subsection


%%% Codebeispiele - Style
\lstdefinestyle{mystyle}{
    basicstyle=\footnotesize,
    aboveskip=10pt,
    breakatwhitespace=false,
    breaklines=true,
    captionpos=b,
    belowcaptionskip=5pt,
    keepspaces=true,
    numbers=left,
    numbersep=15pt,
    showspaces=false,
    showstringspaces=false,
    showtabs=false,
    tabsize=2,
    frame=single,
    lineskip=-0.4pt,
    framexleftmargin=0.5em,
    xleftmargin=2.5em
}
\lstset{style=mystyle}


% Entfernt Kapitel Ueberschrift
% Bsp.
% 	ALT:
%       Kapitel 1
%       Einführung
%
% 	NEU:
% 		1 Einführung
%
\renewcommand*\chapterheadstartvskip{\vspace{-\topskip}}


% Hurenkind und Schusterjunge vermeiden
\clubpenalty = 10000 % schliesst Schusterjungen aus
\widowpenalty = 10000 \displaywidowpenalty = 10000% schliesst Hurenkinder aus


% Alle Listen mit einem "-" statt einem Punkt
\def\labelitemi{--}

