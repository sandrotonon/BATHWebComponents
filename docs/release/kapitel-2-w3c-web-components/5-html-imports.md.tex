\section{HTML Imports}\label{html-imports}

Bisher erlauben es praktisch alle Plattformen, Codeteile zu Importieren
und zu verwenden, nur nicht das Web, bzw. HTML. Das heutige HTML
ermöglicht es externe Stylesheets, JavaScript Dateien, Bilder etc. in
ein HTML Dokument zu importieren, HTML-Dateien selbst können jedoch
nicht importiert werden. Auch ist es nicht möglich, alle benötigten
Dateien in einer Ressource zu bündeln und als einzige Abhängigkeit zu
importieren. HTML Imports versuchen eben dieses Problem zu lösen. So
soll es möglich sein, HTML-Dateien und wiederum HTML-Dateien in
HTML-Dateien zu importieren. So können auch verschiedene benötigte
Dateien in einer HTML-Datei gesammelt und mit nur einem Import in die
Seite eingebunden werden. Doppelte Abhängigkeiten sollen dadurch
automatisch aufgelöst werden, sodass Dateien, die mehrmals eingebunden
werden sollten, automatisch effektiv nur einmal heruntergeladen werden.

\subsection{HTML-Dateien importieren}\label{html-dateien-importieren}

Imports von HTML-Dateien werden, wie andere Imports auch, per
\texttt{\textless{}link\textgreater{}}-Tag deklariert. Neu ist jedoch
der Wert des \texttt{rel}-Attributes, welches auf \texttt{import}
gesetzt wird. \cite[S. 139-147]{citeulike:13844975} Somit kann nun
beispielsweise mittels
\texttt{\textless{}link\ rel="import"\ href="my-import.html"\textgreater{}}
die HTML-Datei \texttt{my-import.html} importiert werden. Sollte nun ein
HTML Import mehrfach aufgeführt sein, oder eine HTML-Datei anfordern,
die schon geladen wurde, so wird die Abhängigkeit automatisch ignoriert
und die Datei nur ein einziges Mal übertragen. Dadurch wird eventuell in
den HTML-Dateien enthaltenes JavaScript auch nur ein mal ausgeführt. Es
ist jedoch zu beachten, dass HTML Imports nur auf Ressourcen der
gleichen Quelle, also dem gleichen Host, respektive der gleichen Domain
zugreifen können. Imports von HTML-Dateien von verschiedenen Quellen
stellen eine Sicherheitslücke dar, da Webbrowser die SOP verfolgen.

\begin{quote}
The same-origin policy restricts how a document or script loaded from
one origin can interact with a resource from another origin. It is a
critical security mechanism for isolating potentially malicious
documents. {[}citeulike:13853253{]}
\end{quote}

Sollte das jedoch dennoch erlaubt werden, so muss das CORS für die
entsprechende Domain auf dem Server aktiviert werden.

\begin{quote}
These restrictions prevent a client-side Web application running from
one origin from obtaining data retrieved from another origin, and also
limit unsafe HTTP requests that can be automatically launched toward
destinations that differ from the running application's origin.
{[}citeulike:13853643{]}
\end{quote}

\subsection{Vorteil}\label{vorteil}

Durch HTML Imports ist es möglich, komplette Applikationen mit mehreren
oder gar nur einer einzigen Anweisung zu importieren. Dies gilt sowohl
für eigene Applikationen oder kleinere Komponenten, wie beispielsweise
einem Slider oder ähnlichem, als auch fremde Frameworks oder
Komponenten. Durch die HTML Imports wird die Abhängigkeiten-Verwaltung
stark vereinfacht und automatisiert.

\begin{quote}
Using only one URL, you can package together a single relocatable bundle
of web goodness for others to consume. {[}citeulike:13853647{]}
\end{quote}

So kann beispielsweise das Bootstrap-Framework statt wie bisher mit
mehreren Imports mit nur einem Import eingebunden werden. Bisher könnte
die Einbindung unter Berücksichtigung der Abhängigkeiten wie folgt
aussehen.

\begin{Shaded}
\begin{Highlighting}[]
\KeywordTok{<link}\OtherTok{ rel=}\StringTok{"stylesheet"}\OtherTok{ href=}\StringTok{"bootstrap.css"}\KeywordTok{>}
\KeywordTok{<link}\OtherTok{ rel=}\StringTok{"stylesheet"}\OtherTok{ href=}\StringTok{"fonts.css"}\KeywordTok{>}
\KeywordTok{<script}\OtherTok{ src=}\StringTok{"jquery.js"}\KeywordTok{></script>}
\KeywordTok{<script}\OtherTok{ src=}\StringTok{"bootstrap.js"}\KeywordTok{></script>}
\KeywordTok{<script}\OtherTok{ src=}\StringTok{"bootstrap-tooltip.js"}\KeywordTok{></script>}
\KeywordTok{<script}\OtherTok{ src=}\StringTok{"bootstrap-dropdown.js"}\KeywordTok{></script>}
\end{Highlighting}
\end{Shaded}

Stattdessen kann dieses Markup nun in ein einziges HTML Dokument,
welches alle Abhängigkeiten verwaltet, geschrieben werden. Dieses wird
dann mit einem einzigen Import in das eigene HTML Dokument importiert.

\begin{Shaded}
\begin{Highlighting}[]
\KeywordTok{<head>}
  \KeywordTok{<link}\OtherTok{ rel=}\StringTok{"import"}\OtherTok{ href=}\StringTok{"bootstrap.html"}\KeywordTok{>}
\KeywordTok{</head>}
\end{Highlighting}
\end{Shaded}

\subsection{HTML Imports verwenden}\label{html-imports-verwenden}

Importierte HTML Dateien werden nicht nur in das Dokument eingefügt,
sondern vom Parser verarbeitet, das bedeutet, dass mit JavaScript auf
das DOM des Imports zugegriffen werden kann. Wenn sie vom Parser
verarbeitet worden sind, sind sie zwar verfügbar, allerdings werden die
Inhalte nicht angezeigt bis sie mittels JavaScript in das DOM eingefügt
werden. Ihre enthaltenen Scripte werden also ausgeführt, Styles und HTML
Knoten werden jedoch ignoriert. Um nun die Inhalte eines Imports auf der
Seite einzubinden und auf sie zuzugreifen muss auf die \texttt{.import}
Eigenschaft des \texttt{\textless{}link\textgreater{}}-Tags mit dem
Import zugegriffen werden
\texttt{var\ content\ =\ document.querySelector(\textquotesingle{}link{[}rel="import"{]}\textquotesingle{}).import;}.
Nun kann mit auf das DOM des Imports zugegriffen werden. Beispielsweise
kann ein enthaltenes Element mit der Klasse \texttt{.element} mit
\texttt{var\ el\ =\ content.querySelector(\textquotesingle{}.element\textquotesingle{});}
geklont und anschließend durch
\texttt{document.body.appendChild(el.cloneNode(true));} in das eigene
DOM eingefügt werden. Falls es nun jedoch mehrere Imports geben sollte,
können den Imports IDs zugewiesen werden, anhand derer die Imports
voneinander unterschieden werden können {[}citeulike:13853724{]}. Der
komplette Prozess wird in dem folgenden Beispiel skizziert.

\begin{Shaded}
\begin{Highlighting}[]
\KeywordTok{<head>}
  \KeywordTok{<link}\OtherTok{ rel=}\StringTok{"import"}\OtherTok{ href=}\StringTok{"my-import.html"}\KeywordTok{>}
\KeywordTok{</head>}
\KeywordTok{<body>}
  \KeywordTok{<script>}
    \KeywordTok{var} \NormalTok{content }\OperatorTok{=} \VariableTok{document}\NormalTok{.}\AttributeTok{querySelector}\NormalTok{(}\StringTok{'link[rel="import"]'}\NormalTok{).}\AttributeTok{import}\OperatorTok{;}
    \KeywordTok{var} \NormalTok{el }\OperatorTok{=} \VariableTok{content}\NormalTok{.}\AttributeTok{querySelector}\NormalTok{(}\StringTok{'.element'}\NormalTok{)}\OperatorTok{;}
    \VariableTok{document}\NormalTok{.}\VariableTok{body}\NormalTok{.}\AttributeTok{appendChild}\NormalTok{(}\VariableTok{el}\NormalTok{.}\AttributeTok{cloneNode}\NormalTok{(}\KeywordTok{true}\NormalTok{))}\OperatorTok{;}
  \OperatorTok{<}\SpecialStringTok{/script>}
\SpecialStringTok{</body}\OperatorTok{>}
\end{Highlighting}
\end{Shaded}

\begin{itemize}
\tightlist
\item
  TODO: Import mit ID machen (mehrere)
\end{itemize}

\subsection{Abhängigkeiten verwalten}\label{abhuxe4ngigkeiten-verwalten}

Doch wie sieht es nun mit Abhängigkeiten von Imports aus? So kommt es
des öfteren vor, dass verschiedene Bibliotheken, welche auf der Seite
eingebunden werden, die gleichen Abhängigkeiten haben und diese
verwaltet werden müssen. Sollte dies nicht sauber gemacht werden, können
unerwartete Fehler auftreten, die mitunter nur schwer zu identifizieren
sind. HTML Imports verwalten diese automatisch. Um das zu erreichen,
wird jede Abhängigkeit in eine zu importierende HTML-Datei geschrieben,
welche diese bündelt. Haben nun mehrere solcher Imports die gleichen
Abhängigkeiten, werden diese vom Browser erkannt und nur einmal
heruntergeladen und eingebunden. Mehrfachdownloads und Konflikte der
Abhängigkeiten werden so verhindert. Das folgende Beispiel von
{[}citeulike:13853700{]} verdeutlicht dieses Szenario.

index.html

\begin{Shaded}
\begin{Highlighting}[]
\KeywordTok{<link}\OtherTok{ rel=}\StringTok{"import"}\OtherTok{ href=}\StringTok{"component1.html"}\KeywordTok{>}
\KeywordTok{<link}\OtherTok{ rel=}\StringTok{"import"}\OtherTok{ href=}\StringTok{"component2.html"}\KeywordTok{>}
\end{Highlighting}
\end{Shaded}

component1.html

\begin{Shaded}
\begin{Highlighting}[]
\KeywordTok{<script}\OtherTok{ src=}\StringTok{"jQuery.html"}\KeywordTok{></script>}
\end{Highlighting}
\end{Shaded}

component2.html

\begin{Shaded}
\begin{Highlighting}[]
\KeywordTok{<script}\OtherTok{ src=}\StringTok{"jQuery.html"}\KeywordTok{></script>}
\end{Highlighting}
\end{Shaded}

jQuery.html

\begin{Shaded}
\begin{Highlighting}[]
\KeywordTok{<script}\OtherTok{ src=}\StringTok{"js/jquery.js"}\KeywordTok{></script>}
\end{Highlighting}
\end{Shaded}

Selbst wenn die jQuery Bibliothek in mehreren Dateien eingebunden wird,
wird hier sie dennoch nur einmal übertragen. Wenn nun fremde
HTML-Dateien eingebunden werden muss auch nicht auf die auf die
Reihenfolge der Imports geachtet werden, da diese selbst ihre
Abhängigkeiten beinhalten.

\subsection{Sub-Imports}\label{sub-imports}

Das obige Beispiel zeigt, dass HTML Imports selbst wiederum auch HTML
Imports beinhalten können. Diese Imports werden Sub-Imports genannt und
ermöglichen einen einfachen Austausch oder eine Erweiterungen von
Abhängigkeiten innerhalb einer Komponente. Wenn eine Komponente A eine
Abhängigkeit von einer Komponenten B hat und eine neue Version von
Komponente B verfügbar ist, kann dies einfach in dem Import des
Sub-Imports angepasst werden ohne den Import in der Eltern-HTML-Datei
anpassen zu müssen {[}citeulike:13853647{]}.

\subsection{Performance}\label{performance}

Ein signifikantes Problem der HTML Imports ist die Performance, welche
bei komplexen Anwendungen schnell abfallen kann. Werden auf einer Seite
mehrere Imports eingebunden, die selbst wiederum Imports beinhalten
können, so steigt die Anzahl an Requests an den Server schnell
exponentiell an, was die Webseite stark verlangsamen kann. Auch das
vorkommen von doppelten Abhängigkeiten kann die Anzahl an Requests in
die Höhe treiben, doch da Browser die Abhängigkeiten nativ schon selbst
de-duplizieren, kann dieses Problem in diesem Zusammenhang ignoriert
werden {[}citeulike:13853714{]}.

\subsection{HTTP/2}\label{http2}

Dennoch sind viele Einzel-Requests ein gewünschtes Verhalten von Web
Components, sie sind also ein Feature und kein Bug. Dieses Feature kommt
jedoch erst zum Tragen, wenn HTTP in Version Zwei in allen Browsern
Standard ist. HTTP/2 bietet eine Reihe an Vorteilen gegenüber dem heute
benutzen HTTP/1.1. So können mehrere Requests in einer TCP-Verbindung
übertragen werden, wobei die Reihenfolge der Antworten auf die Requests
keine Rolle spielt. Des Weiteren kann der Client, also der Absender der
Requests, die Priorität der angeforderten Dateien bestimmen, somit kann
der Server Dateien mit einer hohen Priorität vor Dateien mit niedrigerer
Priorität schicken. Um die Größe der zu sendenden Pakete zu reduzieren,
setzt HTTP/2 eine drastische Header-Kompression ein, welche die
Bruttogröße der zu übertragenden Pakete verkleinert. Neu sind außerdem
die sogenannten ``Server Pushes'', welche es dem Server ermöglichen,
Nachrichten an den Client zu schicken, ohne dass dieser sie anfordern
muss. Durch diese Reihe an neuen Features bietet es also keinen Vorteil
mehr, möglichst wenige Requests an den Server zu schicken um die
Ladezeit der Webseite zu optimieren. HTTP/2 sieht es also vor, seine
Webseite in mehrere kleine Dateien zu unterteilen, sodass bei kleinen
Änderungen einer Datei, wie beispielsweise einem Icon, nicht mehr das
gesamte Bild-Sprite, also das konkatenierte Bild, welches sich aus allen
Einzelbildern zusammensetzt, sondern eben nur das neue, geänderte Icon
neu übertragen werden muss. Ebenso können einzelne Teile einer Webseite
wie Bild-Slider, Navigations-Menüs, komplexe Widgets, etc. schnell
ausgewechselt werden, ohne die komplette Seite bei jeder Anfrage
übertragen zu müssen. Das Zusammenfassen von Dateien wird folglich auf
Protokollebene von HTTP/2 übernommen, die Entwickler einer Webseite
müssen sich nicht selbst um dieses Problem kümmern. Voraussetzung
hierfür ist jedoch die Unterstützung des neuen Protokolls seitens des
Browsers. Bis auf den Internet Explorer und Safari unterstützen jedoch
alle Browser HTTP/2 schon ab frühen Versionen, hier muss allerdings
darauf hingewiesen werden, dass auch dort TLS als Verschlüsselung von
Webseiten verwendet werden muss, damit das HTTP/2 Protokoll benutzt
werden kann {[}citeulike:13879562{]}. Auch wenn das Protokoll schon
standardisiert werden ist, liegt der Market Share an Verbindungen mit
HTTP/2 momentan bei ca. 3\% {[}citeulike:13879575{]}, es muss de facto
darauf verzichtet werden und eine alternative Lösung für das Problem der
vielen Requests gefunden werden.

\subsection{Asynchrones Laden von
Imports}\label{asynchrones-laden-von-imports}

Das Rendern der Seite kann unter Umständen sehr lange dauern, da das in
den Imports enthaltene JavaScript das Rendern blockiert. Um das zu
verhindern und alle Dateien möglichst schnell laden zu können, ist ein
\texttt{async}-Attribut vorgesehen. Dieses funktioniert jedoch nur, wenn
als Protokoll HTTP/2 gewählt wird. Das Attribut ermöglicht es, dass
mehrere Dateien asynchron geladen werden können. In den Imports
enthaltenes JavaScript blockiert somit nicht das Rendern von bereits
geladenem HTML Code. Falls HTTP/2 nicht vorhanden sein sollte, kann
alternativ das \texttt{defer}-Attribut gewählt werden, wodurch das
JavaScipt erst nach vollständigem Parsen des HTML ausgeführt wird. Eine
weitere Methode ist das Laden der Imports am Ende der Seite. Somit wird
sichergestellt, dass die Scripte erst ausgeführt werden, wenn die Seite
geladen und gerendert wurde.

\subsection{\texorpdfstring{Request-Minimierung mit
``Vulcanize''}{Request-Minimierung mit Vulcanize}}\label{request-minimierung-mit-vulcanize}

Webseiten können viele verschiedene, modular aufgebaute Stylesheets,
JavaScript-Dateien etc. beinhalten, welche die Anzahl an Requests
erhöhen. Um die Anzahl an Requests zu verringern gibt es in der
Webentwicklung bereits mehrere verschiedene Hilfsmittel. So werden die
einzelnen Stylesheets oder auch JavaScript Dateien zu einer einzigen
Datei konkateniert, sodass für das komplette Styling und JavaScript
jeweils eine große Datei entsteht, welche nur einen Request an den
Server benötigen. Zusätzlich können die konkatenierten Dateien noch
minifiziert werden, um ihre Größe zu verringern und die Ladezeiten zu
verkürzen. Selbiges Prinzip kann auch auf die HTML Imports angewendet
werden. Google stellt hierfür das Tool Vulcanize
{[}citeulike:13879681{]} bereit, welches serverseitig ermöglicht,
einzelne kleine Web Components eine einzige große Web Component zu
konkatenieren. Benannt nach der Vulkanisation, werden metaphorisch die
einzelnen Elemente in ein beständigere Materialien umgewandelt.
Vulcanize reduziert dabei eine HTML-Datei und ihre zu importierenden
HTML Dateien in eine einzige Datei. Somit werden die unterschiedlichen
Requests in nur einem einzigen Request gebündelt und Ladezeiten und
Bandbreite minimiert.

\subsection{Anwendungen}\label{anwendungen}

Besonders einfach machen HTML Imports das Einbinden ganzer Web
Applikationen mit HTML/JavaScript/CSS. Diese können in eine Datei
geschrieben, ihre Abhängigkeiten definieren und von anderen importiert
werden. Dies macht es sehr einfach, den Code zu organisieren, so können
etwa einzelne Abschnitte einer Anwendung oder von Code in einzelne
Dateien ausgelagert werden, was Web Applikationen modular, austauschbar
und wiederverwendbar macht. Falls nun ein oder mehrere Custom Elements
in einem HTML Import enthalten sind, so werden dessen Interface und
Definitionen automatisch gekapselt. Auch wird die
Abhängigkeitsverwaltung in Betracht auf die Performance stark
verbessert, da der Browser nicht eine große JavaScript-Library, sondern
einzelne kleinere JavaScript-Abschnitte parsen und ausführen muss. Diese
werden durch die HTML Imports parallel geparst, was mit einen enormen
Performance-Schub einher geht {[}citeulike:13853647{]}.

\subsection{Browserunterstützung}\label{browserunterstuxfctzung}

HTML Imports sind noch nicht vom W3C standardisiert, sondern befinden
sich noch im Status eines ``Working Draft'' {[}citeulike:13853711{]}.
Sie werden deshalb bisher nur von Google Chrome ab Version 43 und Opera
ab Version 33 nativ unterstützt. Seitens Mozilla und dessen Browser
Firefox wird es für HTML Imports auch keine Unterstützung geben, da
ihrer Meinung nach der Bedarf an HTML Imports nach der Einführung von
ECMAScript 6 Modules nicht mehr existiert und da Abhängigkeiten schon
mit Tools wie Vulcanize aufgelöst werden können
{[}citeulike:13881144{]}. Auf Details zu ECMAScript 6 Modules wird an
dieser Stelle nicht eingegangen, da diese den Umfang dieser Arbeit
überschreiten.

\begin{figure}[htbp]
\centering
\includegraphics{images/5-html-imports-browserunterstuetzung.jpg}
\caption{Bild: HTML Imports Browserunterstützung}
\end{figure}
